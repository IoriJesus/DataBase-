% run the command ' lualatex -shell-escape Reference.tex ' twice in the terminal to visualize table of contents
\documentclass[twoside]{article}
\usepackage[utf8]{inputenc}
\usepackage[english]{babel}
\usepackage{geometry}
\usepackage{multicol}
\usepackage{minted}
\usepackage{python}
\usepackage[hidelinks]{hyperref}
\usepackage{fancyhdr}
\usepackage{listings}
\usepackage{pdfpages}
\usepackage{needspace}
\usepackage{sectsty}
\usepackage{array}
\usepackage{multirow} 
\usepackage{longtable}
\usepackage{xcolor}
\usepackage{afterpage}
\usepackage{amssymb}
\usepackage{amsmath}
\usepackage[inline]{enumitem}


\newcommand{\wdir}[1]{/home/san/Algorithms/Reference/#1}
\geometry{letterpaper, portrait, left=0.5cm, right=0.5cm, top=1.8cm, bottom=1cm}

\sectionfont{\Huge\bfseries\sffamily}

\definecolor{LightGray}{gray}{0.9}
\definecolor{prussianblue}{rgb}{0.0, 0.19, 0.33}
\definecolor{indigo(dye)}{rgb}{0.0, 0.25, 0.42}
\definecolor{lapislazuli}{rgb}{0.15, 0.38, 0.61}
\definecolor{mediumelectricblue}{rgb}{0.01, 0.31, 0.59}
\definecolor{smalt(darkpowderblue)}{rgb}{0.0, 0.2, 0.6}
\definecolor{yaleblue}{rgb}{0.06, 0.3, 0.57}
\definecolor{skobeloff}{rgb}{0.0, 0.48, 0.45}
\definecolor{pinegreen}{rgb}{0.0, 0.47, 0.44}

\setminted{
    style=tango,
    breaklines=true,
    bgcolor=LightGray
}

\setlength{\headsep}{0.5cm}
\setlength{\columnsep}{0.5cm}
\setlength{\columnseprule}{0.01cm}
\renewcommand{\columnseprulecolor}{\color{gray}}

\pagestyle{fancy}
\pagenumbering{arabic}
\fancyhead{}
\fancyfoot{}
\fancyhead[LO,RE]{\textsf{First, solve the problem. Then, write the code.}}
\fancyhead[LE,RO]{\textsf{\leftmark}}
\fancyfoot[LE,RO]{\textbf{\textsf{\thepage}}}
 
\renewcommand{\headrulewidth}{0.01cm}
\renewcommand{\footrulewidth}{0.01cm}

\setlength{\parindent}{0em}
% column space
\setlength{\tabcolsep}{10pt} % Default value: 6pt
% upper and lower padding
\renewcommand{\arraystretch}{1.5} % Default value: 1

\begin{document}
\null
\thispagestyle{empty}
\newpage
\fontfamily{lmss}
\selectfont
\tableofcontents
\newpage
\cleardoublepage
\section{Introduccion a las bases de datos}
\section{Conceptos de sistemas y arquitectura de bases de datos}
\subsection{DDL Data Definition Language}
Involves:
\begin{itemize}
	\item create table
	\item drop table
\end{itemize}

\subsection{DML Data Manipulation Language}
Involves:
\begin{itemize}
	\item insert into
	\item delete
	\item select
	\item update
\end{itemize}
\section{Algebra relacional y SQL standar}
\textbf{DEF.} Una expresion en el algebra relacional se construye a partir de subexpresiones.\\

Sean $E_1$ \& $E_2$ expresiones del algebra relacional. Las siguientes son todas expresiones \textbf{binarias} del algebra relaional.\\

\begin{itemize}
  \item \textbf{Union:} $E_1 \cup E_2$ (Se requiere que el numero de columnas sea el mismo entre tablas asi como los nombres de columnas.)
  \item \textbf{Diferencia:} $E_1 - E_2$
  \item \textbf{Producto Cartesiano:} $E_1 \times E_2$
\end{itemize}

Y las siguientes son operaciones \textbf{unarias:}\\

\begin{itemize}
  \item \textbf{Seleccion:} $\sigma_p(E_1)$ donde $p$ es un predicado de atributos de $E_1$. (Trabaja con renglones completos)\\
  \item \textbf{Proyeccion:} $\Pi_s(E_1)$ donde $s$ es un conjunto de atributos de $E_1$. (Trabaja con columnas completas)\\
  \item \textbf{Renombramiento:} $P_x(E_1)$ donde $E_x$ es un nuevo nombre de la relacion $E_1$.
\end{itemize}

\newpage

Sean $R$, $S$ y $T$ entidades ejemplificaremos las operaciones mencionadas anteriormente:\\
\begin{center}
\begin{tabular}{ccc}
\multicolumn{3}{c}{\textbf{Entidades}}
\tabularnewline
\begin{minipage}{.25\linewidth}
\begin{tabular}{| >{\centering}m{1cm}| >{\centering}m{1cm}| >{\centering}m{1cm}|}
\hline
\multicolumn{3}{|c|}{\textbf{R}}
\tabularnewline \hline
\textbf{A}
&
\textbf{B}
&
\textbf{C}
\tabularnewline \hline
a & 1 & a
\tabularnewline \hline
b & 1 & b
\tabularnewline \hline
a & 1 & d
\tabularnewline \hline
b & 2 & f
\tabularnewline \hline
\end{tabular}
\end{minipage}
&
\begin{minipage}{.25\linewidth}
\begin{tabular}{| >{\centering}m{1cm}| >{\centering}m{1cm}| >{\centering}m{1cm}|}
\hline
\multicolumn{3}{|c|}{\textbf{S}}
\tabularnewline \hline
\textbf{A}
&
\textbf{B}
&
\textbf{C}
\tabularnewline \hline
a & 1 & a
\tabularnewline \hline
a & 3 & f
\tabularnewline \hline
\end{tabular}
\end{minipage}
&
\begin{minipage}{.25\linewidth}
\begin{tabular}{| >{\centering}m{1cm}| >{\centering}m{1cm}| >{\centering}m{1cm}|}
\hline
\multicolumn{3}{|c|}{\textbf{T}}
\tabularnewline \hline
\textbf{B}
&
\textbf{C}
&
\textbf{D}
\tabularnewline \hline
a & 1 & a
\tabularnewline \hline
3 & b & 1
\tabularnewline \hline
3 & c & 2
\tabularnewline \hline
1 & d & 4
\tabularnewline \hline
2 & a & 3
\tabularnewline \hline
\end{tabular}
\end{minipage}
\end{tabular}
\end{center}


\begin{center}
\begin{tabular}{ccc}
\multicolumn{2}{c}{\textbf{Seleccion}}
\tabularnewline
$SL_{A=a}(R) = \sigma_{A=a}(R)$ \Rightarrow
&
\begin{minipage}{.25\linewidth}
\begin{tabular}{| >{\centering}m{1cm}| >{\centering}m{1cm}| >{\centering}m{1cm}|}
\hline
\multicolumn{3}{|c|}{\textbf{Resultado}}
\tabularnewline \hline
\textbf{A}
&
\textbf{B}
&
\textbf{C}
\tabularnewline \hline
a & 1 & a
\tabularnewline \hline
a & 1 & d
\tabularnewline \hline
\end{tabular}
\end{minipage}
\end{tabular}
\end{center}

\begin{center}
\begin{tabular}{ccc}
\multicolumn{2}{c}{\textbf{Proyeccion}}
\tabularnewline
$PJ_{A,B}(R) = \Pi_{A,B}(R)$ \Rightarrow
&
\begin{minipage}{.25\linewidth}
\begin{tabular}{| >{\centering}m{1cm}| >{\centering}m{1cm}|}
\hline
\multicolumn{2}{|c|}{\textbf{Resultado}}
\tabularnewline \hline
\textbf{A}
&
\textbf{B}
\tabularnewline \hline
a & 1
\tabularnewline \hline
b & 1
\tabularnewline \hline
a & 1
\tabularnewline \hline
b & 2
\tabularnewline \hline
\end{tabular}
\end{minipage}
\end{tabular}
\end{center}

\begin{center}
\begin{tabular}{ccc}
\multicolumn{2}{c}{\textbf{Union}}
\tabularnewline
$Union(R, S) = R \cup S$ \Rightarrow
&
\begin{minipage}{.25\linewidth}
\begin{tabular}{| >{\centering}m{1cm}| >{\centering}m{1cm}| >{\centering}m{1cm}|}
\hline
\multicolumn{3}{|c|}{\textbf{Resultado}}
\tabularnewline \hline
\textbf{A}
&
\textbf{B}
&
\textbf{C}
\tabularnewline \hline
a & 1 & a
\tabularnewline \hline
b & 1 & b
\tabularnewline \hline
a & 1 & d
\tabularnewline \hline
b & 2 & f
\tabularnewline \hline
a & 3 & f
\tabularnewline \hline
\end{tabular}
\end{minipage}
\end{tabular}
\end{center}

\begin{center}
\begin{tabular}{ccc}
\multicolumn{2}{c}{\textbf{Diferencia}}
\tabularnewline
$Dif(R, S) = R - S$ \Rightarrow
&
\begin{minipage}{.25\linewidth}
\begin{tabular}{| >{\centering}m{1cm}| >{\centering}m{1cm}| >{\centering}m{1cm}|}
\hline
\multicolumn{3}{|c|}{\textbf{Resultado}}
\tabularnewline \hline
\textbf{A}
&
\textbf{B}
&
\textbf{C}
\tabularnewline \hline
b & 1 & b
\tabularnewline \hline
a & 1 & d
\tabularnewline \hline
b & 2 & f
\tabularnewline \hline
\end{tabular}
\end{minipage}
\end{tabular}
\end{center}

\begin{center}
\begin{tabular}{ccc}
\multicolumn{2}{c}{\textbf{Producto Cartesiano}}
\tabularnewline
$PC(R, S) = R \times S$ \Rightarrow
&
\begin{minipage}{.25\linewidth}
\begin{tabular}{| >{\centering}m{1cm}| >{\centering}m{1cm}| >{\centering}m{1cm}| >{\centering}m{1cm}| >{\centering}m{1cm}| >{\centering}m{1cm}|}
\hline
\multicolumn{6}{|c|}{\textbf{Resultado}}
\tabularnewline \hline
\textbf{R.A}
&
\textbf{R.B}
&
\textbf{R.C}
&
\textbf{S.A}
&
\textbf{S.B}
&
\textbf{S.C}
\tabularnewline \hline
a & 1 & a & a & 1 & a
\tabularnewline \hline
b & 1 & b & a & 1 & a
\tabularnewline \hline
a & 1 & d & a & 1 & a
\tabularnewline \hline
b & 2 & f & a & 1 & a
\tabularnewline \hline
a & 1 & a & a & 3 & f
\tabularnewline \hline
b & 1 & b & a & 3 & f
\tabularnewline \hline
a & 1 & d & a & 3 & f
\tabularnewline \hline
b & 2 & f & a & 3 & f
\tabularnewline \hline
\end{tabular}
\end{minipage}
\end{tabular}
\end{center}

\begin{center}
\begin{tabular}{ccc}
\multicolumn{2}{c}{\textbf{Producto Cartesiano}}
\tabularnewline
$PC(R, T) = R \times T$ \Rightarrow
&
\begin{minipage}{.25\linewidth}
\begin{tabular}{| >{\centering}m{1cm}| >{\centering}m{1cm}| >{\centering}m{1cm}| >{\centering}m{1cm}| >{\centering}m{1cm}| >{\centering}m{1cm}|}
\hline
\multicolumn{6}{|c|}{\textbf{Resultado}}
\tabularnewline \hline
\textbf{R.A}
&
\textbf{R.B}
&
\textbf{T.B}
&
\textbf{R.C}
&
\textbf{T.C}
&
\textbf{T.D}
\tabularnewline \hline
a & 1 & 1 & a & a & 1
\tabularnewline \hline
b & 1 & 1 & b & a & 1
\tabularnewline \hline
a & 1 & 1 & d & a & 1
\tabularnewline \hline
b & 2 & 1 & f & a & 1
\tabularnewline \hline
a & 1 & 3 & a & b & 1
\tabularnewline \hline
b & 1 & 3 & b & b & 1
\tabularnewline \hline
a & 1 & 3 & d & b & 1
\tabularnewline \hline
b & 2 & 3 & f & b & 1
\tabularnewline \hline
a & 1 & 3 & a & c & 2
\tabularnewline \hline
b & 1 & 3 & b & c & 2
\tabularnewline \hline
a & 1 & 3 & d & c & 2
\tabularnewline \hline
b & 2 & 3 & f & c & 2
\tabularnewline \hline
a & 1 & 1 & a & d & 4
\tabularnewline \hline
b & 1 & 1 & b & d & 4
\tabularnewline \hline
a & 1 & 1 & d & d & 4
\tabularnewline \hline
b & 2 & 1 & f & d & 4
\tabularnewline \hline
a & 1 & 2 & a & a & 3
\tabularnewline \hline
b & 1 & 2 & b & a & 3
\tabularnewline \hline
a & 1 & 2 & d & a & 3
\tabularnewline \hline
b & 2 & 2 & f & a & 3
\tabularnewline \hline
\end{tabular}
\end{minipage}
\end{tabular}
\end{center}

\begin{center}
\begin{tabular}{ccc}
\multicolumn{2}{c}{\textbf{JOIN}}
\tabularnewline
$JOIN(R, T)_{R.C = T.C}$ \Rightarrow
&
\begin{minipage}{.25\linewidth}
\begin{tabular}{| >{\centering}m{1cm}| >{\centering}m{1cm}| >{\centering}m{1cm}| >{\centering}m{1cm}| >{\centering}m{1cm}| >{\centering}m{1cm}|}
\hline
\multicolumn{6}{|c|}{\textbf{Resultado}}
\tabularnewline \hline
\textbf{R.A}
&
\textbf{R.B}
&
\textbf{R.C}
&
\textbf{T.B}
&
\textbf{T.C}
&
\textbf{T.D}
\tabularnewline \hline
a & 1 & a & 1 & a & 1
\tabularnewline \hline
b & 1 & b & 3 & b & 1
\tabularnewline \hline
a & 1 & d & 1 & d & 4
\tabularnewline \hline
a & 1 & a & 2 & a & 3
\tabularnewline \hline
\end{tabular}
\end{minipage}
\end{tabular}
\end{center}

\section{Analisis de una base de datos}
\section{Diseno de una base de datos}


\section{Oracle DB Commands}

\subsection{Start Oracle DB}

\begin{minted}{console}
[root@localhost ~]# . ./setupXEvars
\end{minted}
The contents of this script are the following:

\begin{minted}{bash}
export ORACLE_SID=XE
export ORAENV_ASK=NO
. /opt/oracle/product/18c/dbhomeXE/bin/oraenv
/sbin/service oracle-xe-18c start
\end{minted}

\subsection{Login as sysdba (root)}
\begin{minted}{console}
[root@localhost ~]# sqlplus sys as sysdba
\end{minted}
\subsection{Alter Oracle Date Language}
\begin{minted}{sql}
alter session set nls_session_parameters
alter session set nls_date_language='english'
select * from nls_session_parameters;
\end{minted}
\subsection{Create User}
Our username will be "sergio" and our password will be also "sergio"; \\
Login as sysdba and enter the following commands.
\begin{minted}{sql}
alter session set "_ORACLE_SCRIPT"=true;
create user sergio identified by sergio default tablespace users temporary tablespace temp;
grant connect, resource to sergio;
alter user sergio quota unlimited on users;
alter user sergio quota unlimited on temp;
\end{minted}
\subsection{Create Student Database from script}
\begin{minted}{sql}
SQL> @dbs/O11/createStudent.sql
\end{minted}
\subsection{SPOOL Command}
\begin{minted}{sql}
Saves all the output of the inserted commands into the given file (mylog.txt)
SQL> spool mylog.txt
\end{minted}
\subsection{TO\_DATE() Function}
\begin{minted}{sql}
to_date('05-apr-2003 20:14:33', 'dd-mon-yyyy hh24:mm:ss');
\end{minted}

\subsection{ROWNUM Column}
\textbf{ROWNUM: } Is a column appended to every table which corresponds to the index of every row in the table.
\subsection{Install Sample Database}
Login as sysdba and run the following commands.
\begin{minted}{sql}
alter session set "_ORACLE_SCRIPT"=true;
@?/demo/schema/human_resources/hr_main.sql
\end{minted}
\section{Bibliografia}

SQL by Example

\section{Tareas}
\subsection{Investigar que es SQL [Done]}
SQL stands for "Structure Query Language" and is a language used to perform queries in databases managers.
\subsection{Investigar que es "SPOOL" [Done]}
Is a command used to save the ouput of the commands we insert into the sqlplus console.
\section{Student Database Schema}
\includepdf{AppendixD.pdf}

\end{document}
