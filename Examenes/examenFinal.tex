% run the command ' lualatex -shell-escape Reference.tex ' twice in the terminal to visualize table of contents
\documentclass[twoside]{article}
\usepackage[utf8]{inputenc}
\usepackage[english]{babel}
\usepackage{geometry}
\usepackage{multicol}
\usepackage{minted}
\usepackage{python}
\usepackage[hidelinks]{hyperref}
\usepackage{fancyhdr}
\usepackage{listings}
\usepackage{pdfpages}
\usepackage{needspace}
\usepackage{sectsty}
\usepackage{array}
\usepackage{multirow}
\usepackage{longtable}
\usepackage{xcolor}
\usepackage{afterpage}
\usepackage{amssymb}
\usepackage{amsmath}
\usepackage[inline]{enumitem}
\usepackage{fontspec}

\geometry{letterpaper, portrait, left=1.5cm, right=1.5cm, top=2.5cm, bottom=2.5cm}

\setminted{
    style=tango,
    breaklines=true
}

\setlength{\headsep}{0.5cm}
\setlength{\columnsep}{0.5cm}
\setlength{\columnseprule}{0.01cm}
\renewcommand{\columnseprulecolor}{\color{gray}}

\pagestyle{fancy}
\pagenumbering{arabic}
\fancyhead{}
\fancyfoot{}
\fancyhead[LO,RE]{\textsf{Autor: Sergio Gabriel Sanchez Valencia}}
\fancyhead[LE,RO]{\textsf{\leftmark}}
\fancyfoot[LE,RO]{\textbf{\textsf{\thepage}}}

\renewcommand{\headrulewidth}{0.01cm}
\renewcommand{\footrulewidth}{0.01cm}

\setlength{\parindent}{0em}
% column space
\setlength{\tabcolsep}{10pt} % Default value: 6pt
% upper and lower padding
\renewcommand{\arraystretch}{1.5} % Default value: 1

\definecolor{prussianblue}{rgb}{0.0, 0.19, 0.33}
\definecolor{indigo(dye)}{rgb}{0.0, 0.25, 0.42}
\definecolor{lapislazuli}{rgb}{0.15, 0.38, 0.61}
\definecolor{mediumelectricblue}{rgb}{0.01, 0.31, 0.59}
\definecolor{smalt(darkpowderblue)}{rgb}{0.0, 0.2, 0.6}
\definecolor{yaleblue}{rgb}{0.06, 0.3, 0.57}
\definecolor{skobeloff}{rgb}{0.0, 0.48, 0.45}
\definecolor{pinegreen}{rgb}{0.0, 0.47, 0.44}

\begin{document}
\section*{Respuestas a Examen}
\begin{enumerate}
      \item
            \begin{itemize}
                  \item \textbf{Referencia DBA:} ''El DBA es responsable de autorizar el acceso a la base de datos, coordinando y monitoreando su uso,
                        y adquiriendo recursos necesarios de software y hardware.'' (Elmasri, Navathe, 2011, pp. 15)
                  \item \textbf{Respuesta DBA:} Basicamente el DBA es el encargado de otorgar los accesos correspondientes a la base de datos, ademas de coordinar y monitorear
                        el uso de la misma. Adem\'as, este se encarga de adquirir el software y hardware necesario para mantener el funcionamiento adecuado de la base de datos.
                  \item \textbf{Referencia DBD:} ''Los diseñadores de bases de datos son responsables de identificar los datos que deben ser almacenados en una base de datos
                        y de elegir las estruturas apropiadas para representar y guardar estos datos'' (Elmasri, Navathe, 2011, pp. 15)
                  \item \textbf{Respuesta DBD:} Los diseñadores de bases de datos tienen la responsabilidad de analizar los datos y deducir cuales de estos deben ser almacenados
                        en la base de datos, adem\'as de esto, deben de proporcionar la forma en la que estos datos se almacenaran.
            \end{itemize}
      \item
            \begin{itemize}
                  \item \textbf{Referencia integridad de entidad:} ''La restricci\'on de integridad de entidad establece que ninguna llave primaria puede ser NULA.'' (Elmasri, Navathe, 2011, pp. 73)
                  \item \textbf{Referencia integridad referencial:} ''La restricci\'on de integridad referencial esta especificada entre dos realciones y es utilizada
                        para mantener la consistencia entre tuplas en las dos relaciones.'' (Elmasri, Navathe, 2011, pp. 73)
                  \item \textbf{Respuesta:} Considero que el uso de la restricci\'on de integridad de entidad es muy importante ya que nos ayuda a identificar de manera \'unica a cada tupla en la enidad.
                        Y en el mismo nivel de importancia se encuentra la restriccion de integridad referencial ya que con esta se establece que ambas tuplas relacionadas deben existir en las dos entidades.
                        Finalmente, el aplicar estas dos restricciones es muy importante ya que garantizan que las conexiones entre entidades se mantengan su integridad.
            \end{itemize}
      \item
            \textbf{Operaciones Unarias:}
            \begin{itemize}
                  \item \textbf{Referencia SELECT:} ''La operaci\'on SELECT es usada para elegir un subconjunto de tuplas the una relaci\'on que satisfaga una condici\'on de selecci\'on.''
                        (Elmasri, Navathe, 2011, pp. 147)
                  \item \textbf{Respuesta SELECT:} Considero que la operaci\'on SELECT es un filtro que mantiene solo aquellas tuplas que cumplen con ciertas condiciones.
                  \item \textbf{Referencia PROJECT:} ''La operaci\'on PROJECT selecciona ciertas columnas de una tabla y descarta las demas columnas'' (Elmasri, Navathe, 2011, pp. 149)
                  \item \textbf{Respuesta PROJECT:} La operaci\'on PROJECT se utiliza cuando solo queremos conocer el valor de ciertos atributos en una relaci\'on ya que con esta le indicamos que atributos proyectar.
                  \item \textbf{Referencia RENAME:} ''Puede renombrar ya sea el nombre de la relaci\'on \'o los nombres de los atributos, \'o ambos como un operador unario''
                        (Elmasri, Navathe, 2011, pp. 152)
                  \item \textbf{Respuesta RENAME:} La operaci\'on de renombramiento nos ayuda para que nuestras consultas puedan ser entendidas de manera mas simple y adem\'as simplificar las mismas.
            \end{itemize}
            \textbf{Operaciones Binarias:}
            \begin{itemize}
                  \item \textbf{Referencia UNI\'ON:} ''El resultado de esta operaci\'on, denotado por $R \cup S$, es una relaci\'on que incluye todas las tuplas que est\'an
                        \'o en $R$ \'o en $S$ \'o en ambas $R$ y $S$. Tuplas duplicadas son eliminadas.'' (Elmasri, Navathe, 2011, pp. 153)
                  \item \textbf{Respuesta UNI\'ON:} La operaci\'on UNI\'ON nos ayuda a unir las tuplas de dos o mas relaciones.
                  \item \textbf{Referencia INTERSECCI\'ON:} ''El resultado de esta operacio\'on, denotado por $R \cap S$, es una relaci\'on que incluye todas las tuplas
                        que est\'an en ambas $R$ y $S$.'' (Elmasri, Navathe, 2011, pp. 153)
                  \item \textbf{Respuesta INTERSECCI\'ON:} La operaci\'on INTERSECCI\'ON selecciona solo aquellas tuplas que aparecen en ambas relaciones.
                  \item \textbf{Referencia DIFERENCIA:} ''El resultado de esta operaci\'on, denotado por $R - S$, es una relaci\'on que incluye todas las tuplas que est\'an
                        en $R$ pero no en $S$.'' (Elmasri, Navathe, 2011, pp. 153)
                  \item \textbf{Respuesta DIFERENCIA:} La operaci\'on DIFERENCIA elimina de $R$ todas las tuplas que comparta con $S$.
            \end{itemize}
            \textbf{Referencia Compatibilidad a la uni\'on:} ''Dos relaciones $R(A_1, A_2, \cdots, A_n)$ y $S(B_1, B_2, \cdots, B_n)$ se dicen compatibles a la uni\'on si
            tienen el mismo grado $n$ y si $dom(A_i) = dom(B_i)$ para $1 \leq i \leq n$.'' (Elmasri, Navathe, 2011, pp. 153)\\
            \textbf{Respuesta Compatibilidad a la uni\'on:} Basicamente esto significa que las dos relaciones deben tener el mismo n\'umero de atributos y cada par correspondiente
            tiene el mismo dominio.
      \item
            \textbf{Anomal\'ias:}
            \begin{itemize}
                  \item \textbf{Referencia Anomal\'ia 1:} ''Desperdicio de espacio debido a NULOS y la dificultad de ejecutar operaciones de selecci\'on, agregaci\'on y uniones debido a los valores NULOS.''
                        (Elmasri, Navathe, 2011, pp. 513)
                  \item \textbf{Referencia Anomal\'ia 2:} ''Generaci\'on the datos invalidos y falsos durante las uniones en relaciones con atributos correspondientes que puede que no representen una relaci\'on apropiada.''
                        (Elmasri, Navathe, 2011, pp. 513)
                  \item \textbf{Ejemplo Anomal\'ia 1:} Si tenemos una relaci\'on en la cual su identificador es NULO entonces al hacer la uni\'on con otra relaci\'on considerando los identificadores entonces se realizarian
                        uniones que no tendr\'ian sentido.
                  \item \textbf{Ejemplo Anomal\'ia 2:} Si tenemos alguna relaci\'on con alg\'un atributo con posibles valores nulos esto significa que una tupla puede tener o no tener cierto atributo.
                        Lo que nos lleva a tener una columna que desperdicia espacios para las tuplas que no tienen dicho atributo. Se podr\'ia crear una nueva relaci\'on en su lugar para evitar esto.
            \end{itemize}
      \item \textbf{Referencia de Dependencia Funcional:} ''Una dependencia funcional es una restricci\'on entre dos conjuntos de atributos de la base de datos.'' (Elmasri, Navathe, 2011, pp. 513)\\
            \textbf{Respuesta:} Basicamente una dependencia funcional nos indica que cierto atributo $Y$ es determinado por otro atributo $X$ y se denota como $X \rightarrow Y$,
            lo que significa que en todas las tuplas que exista $X_i$ entonces aparecera $Y_j$.
      \item
            \begin{itemize}
                  \item \textbf{Referencia 1NF:} ''Establece que el dominio de un atributo debe incluir solo valores at\'omicos y que el valor de cada atributo en una tupla
                        debe ser un solo valor del dominio de ese atributo'' (Elmasri, Navathe, 2011, pp. 519)
                  \item \textbf{Respuesta 1NF:} Un esquema esta en 1FN si no contiene atributos multivalor, es decir todos los atributos deben ser atómicos (indivisibles).
                  \item \textbf{Referencia 2NF:} ''Un esquema relacional $R$ se encuentra en 2NF si todo atributo no primario $A$ en $R$ es funcionalemente dependiente en su totalidad de la
                        llave primaria de $R$.'' (Elmasri, Navathe, 2011, pp. 523)
                  \item \textbf{Respuesta 2NF:}  Un esquema esta en 2FN si todos los atributos son completamente dependientes de la llave primaria.
                  \item \textbf{Referencia 3NF:} ''Seg\'un la definici\'on original de Codd, un esquema relacional $R$ se encuentra en 3NF is satisface la 2NF y ningun atributo no primario de $R$
                        es transitivamente dependiente de la llave primaria'' (Elmasri, Navathe, 2011, pp. 524)
                  \item \textbf{Respuesta 3NF:} Un esquema esta en 3FN si satisface 2FN y además ningún atributo no primario depende de manera transitiva de la llave primaria. Es decir todos
                        los atributos de la relaci\'on dependen solo de la llave primaria de manera directa y no atraves de otro atributos.
            \end{itemize}
      \item
            \begin{itemize}
                  \item \textbf{Referencia BCNF:} ''Un esquema relacional $R$ se encuentra en BCNF si para cada dependencia funcional no trivial $X \rightarrow A$ que existe en $R$,
                        entonces $X$ es una superllave de $R$'' (Elmasri, Navathe, 2011, pp. 529)
                  \item \textbf{Respuesta BCNF:} Un esquema esta en BCNF si para todas las dependencias no triviales ''X \rightarrow A'', ''X'' es una superllave de la relación.
            \end{itemize}
      \item
            \begin{itemize}
                  \item \textbf{Referencia DF Multivaluada:} ''Una dependencia multivaluada $X \rightarrow \rightarrow Y$ especificada en el esquema $R$, donde $X$ y $Y$ son ambos
                        subconjuntos de $R$, especifica la siguiente restricci\'on en todos los estados $r$ de la relaci\'on $R$: Si dos tuplas $t_1$ y $t_2$ existen en $r$ tal que
                        $t_1[X] = t_2[X]$, entonces dos tuplas $t_3$ y $t_4$ tambi\'en deben existir en $r$ con las siguientes propiedades, donde usamos $Z$ para denotar $(R - (X \cup Y))$:
                        $t_3[X] = t_4[X] = t_1[X] = t_2[X]$, $t_3[Y] = t_1[Y]$ y $t_4[Y] = t_2[Y]$, $t_3[Z] = t_2[Z]$ y $t_4[Z] = t_1[Z]$.'' (Elmasri, Navathe, 2011, pp. 533)
                  \item \textbf{Respuesta DF Multivaluada:}  Las dependencias multivaluadas se generan cuando un atributo $A$ determina a otro, pero este no siempre tiene los mismos valores
                        debido a $A$.
            \end{itemize}
      \item
            \begin{itemize}
                  \item \textbf{Referencia a dependencia reuni\'on:} ''Denotado por $JD(R_1, R_2, \cdots, R_n)$, especificado en el esquema de relaci\'on $R$, especifica una restricci\'on
                        en el esquema relacional $R$. La restricci\'on establece que para todos los estados legales $r$ de $R$ deben tener una descomposici\'on a la uni\'on no aditiva en
                        $R_1, R_2, \cdots, R_n$. Por lo tanto, para cada $r$ tenemos $*(\pi_{R_1}(r), \pi_{R_2}(r), \cdots , \pi_{R_n}(r) = r$.'' (Elmasri, Navathe, 2011, pp. 535)
                  \item \textbf{Respuesta a dependencia reuni\'on:} Basicamente nos indica que al hacer una descomposici\'on esta no debe tener perdidas al hacer la reuni\'on.
                  \item \textbf{Referencia a Quinta Forma Normal:} ''Un esquema relacional $R$ se encuentra en quinta forma normal (5NF) con respecto al conjunto $F$ de dependencias
                        funcionales, multivaluadas y de uni\'on si, para toda dependencia de uni\'on no trivial $JD(R_1, R_2, \cdots , R_n)$ en $F^+$, cada $R_i$ es una superllave de $R$.''
                        (Elmasri, Navathe, 2011, pp. 535)
                  \item \textbf{Respuesta a Quinta Forma Normal:} Una relación esta en quinta forma normal (5FN) si esta se encuentra en 4FN y las únicas dependencias que existen
                        son las denominadas dependencias de Join de una tabla con sus proyecciones, relacionándose entre sí mediante la clave primaria, o cualquier clave alternativa.
            \end{itemize}
      \item
            \begin{itemize}
                  \item \textbf{Referencia propiedad cerradura:} ''Formalmente, el conjunto de todas las dependencias que incluyen $F$ asi como a las dependencias que pueden ser inferidas
                        the $F$ es llamada la cerradura de $F$; se denota como $F^+$.'' (Elmasri, Navathe, 2011, pp. 545)
                  \item \textbf{Respuesta propiedad cerradura:} Siendo $F$ un conjunto de dependencias funcionales, entonces la cerradura de $F$ ($F^+$) es igual a $F$ m\'as las dependencias
                        funcionales que se puedan derivar de $F$.
                  \item \textbf{Ejemplo:} 
            \end{itemize}
      \item
            \begin{itemize}
                  \item \textbf{Referencia de conjuntos equivalentes:} ''Dos conjuntos de dependencias funcionales $E$ y $F$ son equivalentes si $E+ = F+$.'' (Elmasri, Navathe, 2011, pp. 549)
                  \item \textbf{Respuesta de conjuntos esquivalentes:} Esto significa que todas las DF en $E$ pueden ser inferidas de $F$ y que todas las dependencias en $F$ pueden ser
                        inferidas de $E$. y podemos calcular esto calculando que $E$ cubre $F$ y que $F$ cubre a $E$.
            \end{itemize}
      \item
            \begin{itemize}
                  \item \textbf{Referencia de cubierta minima:} ''Una cubierta minima de un conjunto de dependencias funcionales $E$ es un conjunto minimo de dependencias (en
                        la forma can\'onica estandar y sin redundancia) que es equivalente a $E$.'' (Elmasri, Navathe, 2011, pp. 550)
                  \item \textbf{Resputesta de cubierta minima:} Dado un conjunto de dependencias funcionales que tiene redundancia y no esta en su forma canonica entonces su cubierta minima
                        es el mismo conjunto eliminando redundancias y transformandolo a su forma can\'onica.
            \end{itemize}
      \item \textbf{Respuesta de preservaci\'on de atributos en la descomposici\'on:} Esta condici\'on establece que al descomponer una relaci\'on $R$ en $R' = R_1, R_2, \cdots, R_n$ entonces
            al efectuar la reuni\'on se deben de poder recuperar los mismos atributos que fueron descompuestos.
      \item
            \begin{itemize}
                  \item \textbf{Referencia de insuficiencia de las formas normales:} ''Debemos de considerar la descomposici\'on de la relaci\'on universal como un todo, adem\'as de analizar las
                        relaciones individuales.'' (Elmasri, Navathe, 2011, pp. 552)
                  \item \textbf{Respuesta de insuficiencia de las formas normales:} No es suficiente transformar las relaciones individuales en su forma normal, por que existe la posibilidad
                        de que al hacer la reuni\'on de las relaciones descompuestas se generen tuplas espurias. Y es por esto que debemos analizar todas las descomposiciones como un todo. Para
                        asi garantizar la integridad en las consultas.
            \end{itemize}
\end{enumerate}
\includepdf{"Examen_final_bases.pdf"}
\end{document}

